\PassOptionsToPackage{unicode=true}{hyperref} % options for packages loaded elsewhere
\PassOptionsToPackage{hyphens}{url}
%
\documentclass[]{article}
\usepackage{lmodern}
\usepackage{amssymb,amsmath}
\usepackage{ifxetex,ifluatex}
\usepackage{fixltx2e} % provides \textsubscript
\ifnum 0\ifxetex 1\fi\ifluatex 1\fi=0 % if pdftex
  \usepackage[T1]{fontenc}
  \usepackage[utf8]{inputenc}
  \usepackage{textcomp} % provides euro and other symbols
\else % if luatex or xelatex
  \usepackage{unicode-math}
  \defaultfontfeatures{Ligatures=TeX,Scale=MatchLowercase}
\fi
% use upquote if available, for straight quotes in verbatim environments
\IfFileExists{upquote.sty}{\usepackage{upquote}}{}
% use microtype if available
\IfFileExists{microtype.sty}{%
\usepackage[]{microtype}
\UseMicrotypeSet[protrusion]{basicmath} % disable protrusion for tt fonts
}{}
\IfFileExists{parskip.sty}{%
\usepackage{parskip}
}{% else
\setlength{\parindent}{0pt}
\setlength{\parskip}{6pt plus 2pt minus 1pt}
}
\usepackage{hyperref}
\hypersetup{
            pdftitle={Big Data na Prática 4 - Customer Churn Analytics},
            pdfborder={0 0 0},
            breaklinks=true}
\urlstyle{same}  % don't use monospace font for urls
\usepackage[margin=1in]{geometry}
\usepackage{graphicx,grffile}
\makeatletter
\def\maxwidth{\ifdim\Gin@nat@width>\linewidth\linewidth\else\Gin@nat@width\fi}
\def\maxheight{\ifdim\Gin@nat@height>\textheight\textheight\else\Gin@nat@height\fi}
\makeatother
% Scale images if necessary, so that they will not overflow the page
% margins by default, and it is still possible to overwrite the defaults
% using explicit options in \includegraphics[width, height, ...]{}
\setkeys{Gin}{width=\maxwidth,height=\maxheight,keepaspectratio}
\setlength{\emergencystretch}{3em}  % prevent overfull lines
\providecommand{\tightlist}{%
  \setlength{\itemsep}{0pt}\setlength{\parskip}{0pt}}
\setcounter{secnumdepth}{0}
% Redefines (sub)paragraphs to behave more like sections
\ifx\paragraph\undefined\else
\let\oldparagraph\paragraph
\renewcommand{\paragraph}[1]{\oldparagraph{#1}\mbox{}}
\fi
\ifx\subparagraph\undefined\else
\let\oldsubparagraph\subparagraph
\renewcommand{\subparagraph}[1]{\oldsubparagraph{#1}\mbox{}}
\fi

% set default figure placement to htbp
\makeatletter
\def\fps@figure{htbp}
\makeatother


\title{Big Data na Prática 4 - Customer Churn Analytics}
\author{}
\date{\vspace{-2.5em}}

\begin{document}
\maketitle

\begin{verbatim}
## 'data.frame':    7043 obs. of  21 variables:
##  $ customerID      : Factor w/ 7043 levels "0002-ORFBO","0003-MKNFE",..: 5376 3963 2565 5536 6512 6552 1003 4771 5605 4535 ...
##  $ gender          : Factor w/ 2 levels "Female","Male": 1 2 2 2 1 1 2 1 1 2 ...
##  $ SeniorCitizen   : int  0 0 0 0 0 0 0 0 0 0 ...
##  $ Partner         : Factor w/ 2 levels "No","Yes": 2 1 1 1 1 1 1 1 2 1 ...
##  $ Dependents      : Factor w/ 2 levels "No","Yes": 1 1 1 1 1 1 2 1 1 2 ...
##  $ tenure          : int  1 34 2 45 2 8 22 10 28 62 ...
##  $ PhoneService    : Factor w/ 2 levels "No","Yes": 1 2 2 1 2 2 2 1 2 2 ...
##  $ MultipleLines   : Factor w/ 3 levels "No","No phone service",..: 2 1 1 2 1 3 3 2 3 1 ...
##  $ InternetService : Factor w/ 3 levels "DSL","Fiber optic",..: 1 1 1 1 2 2 2 1 2 1 ...
##  $ OnlineSecurity  : Factor w/ 3 levels "No","No internet service",..: 1 3 3 3 1 1 1 3 1 3 ...
##  $ OnlineBackup    : Factor w/ 3 levels "No","No internet service",..: 3 1 3 1 1 1 3 1 1 3 ...
##  $ DeviceProtection: Factor w/ 3 levels "No","No internet service",..: 1 3 1 3 1 3 1 1 3 1 ...
##  $ TechSupport     : Factor w/ 3 levels "No","No internet service",..: 1 1 1 3 1 1 1 1 3 1 ...
##  $ StreamingTV     : Factor w/ 3 levels "No","No internet service",..: 1 1 1 1 1 3 3 1 3 1 ...
##  $ StreamingMovies : Factor w/ 3 levels "No","No internet service",..: 1 1 1 1 1 3 1 1 3 1 ...
##  $ Contract        : Factor w/ 3 levels "Month-to-month",..: 1 2 1 2 1 1 1 1 1 2 ...
##  $ PaperlessBilling: Factor w/ 2 levels "No","Yes": 2 1 2 1 2 2 2 1 2 1 ...
##  $ PaymentMethod   : Factor w/ 4 levels "Bank transfer (automatic)",..: 3 4 4 1 3 3 2 4 3 1 ...
##  $ MonthlyCharges  : num  29.9 57 53.9 42.3 70.7 ...
##  $ TotalCharges    : num  29.9 1889.5 108.2 1840.8 151.7 ...
##  $ Churn           : Factor w/ 2 levels "No","Yes": 1 1 2 1 2 2 1 1 2 1 ...
\end{verbatim}

Os dados brutos contém 7043 linhas (clientes) e 21 colunas (recursos). A
coluna ``Churn'' é o nosso alvo.

\begin{verbatim}
##       customerID           gender    SeniorCitizen          Partner 
##                0                0                0                0 
##       Dependents           tenure     PhoneService    MultipleLines 
##                0                0                0                0 
##  InternetService   OnlineSecurity     OnlineBackup DeviceProtection 
##                0                0                0                0 
##      TechSupport      StreamingTV  StreamingMovies         Contract 
##                0                0                0                0 
## PaperlessBilling    PaymentMethod   MonthlyCharges     TotalCharges 
##                0                0                0               11 
##            Churn 
##                0
\end{verbatim}

\begin{enumerate}
\def\labelenumi{\arabic{enumi}.}
\item
  Vamos mudar ``No internet service'' para ``No'' por seis colunas, que
  são: ``OnlineSecurity'', ``OnlineBackup'', ``DeviceProtection'',
  ``TechSupport'', ``streamingTV'', ``streamingMovies''.
\item
  Vamos mudar ``No phone service'' para ``No'' para a coluna
  ``MultipleLines''
\end{enumerate}

Como a permanência mínima é de 1 mês e a permanência máxima é de 72
meses, podemos agrupá-los em cinco grupos de posse (tenure): ``0-12
Mês'', ``12--24 Mês'', ``24--48 Meses'', ``48--60 Mês'' Mês
'',``\textgreater{} 60 Mês''

\begin{verbatim}
## [1] 1
\end{verbatim}

\begin{verbatim}
## [1] 72
\end{verbatim}

Alteramos os valores na coluna ``SeniorCitizen'' de 0 ou 1 para ``No''
ou ``Yes''.

Removemos as colunas que não precisamos para a análise.

\#\#Análise exploratória de dados e seleção de recursos

\includegraphics{07-Big-Data-na-Pratica4_files/figure-latex/unnamed-chunk-12-1.pdf}

Os encargos mensais e os encargos totais estão correlacionados.

\hypertarget{gruxe1ficos-de-barra-de-variuxe1veis-categuxf3ricas}{%
\subsection{Gráficos de barra de variáveis
categóricas}\label{gruxe1ficos-de-barra-de-variuxe1veis-categuxf3ricas}}

\includegraphics{07-Big-Data-na-Pratica4_files/figure-latex/unnamed-chunk-14-1.pdf}

\includegraphics{07-Big-Data-na-Pratica4_files/figure-latex/unnamed-chunk-15-1.pdf}

\includegraphics{07-Big-Data-na-Pratica4_files/figure-latex/unnamed-chunk-16-1.pdf}

\includegraphics{07-Big-Data-na-Pratica4_files/figure-latex/unnamed-chunk-17-1.pdf}

Todas as variáveis categóricas parecem ter uma distribuição
razoavelmente ampla, portanto, todas elas serão mantidas para análise
posterior.

\hypertarget{regressuxe3o-loguxedstica}{%
\subsection{Regressão Logística}\label{regressuxe3o-loguxedstica}}

Primeiro, dividimos os dados em conjuntos de treinamento e testes

Confirme se a divisão está correta

\begin{verbatim}
## [1] 4924   19
\end{verbatim}

\begin{verbatim}
## [1] 2108   19
\end{verbatim}

Treinando o modelo de regressão logística

\begin{verbatim}
## 
## Call:
## glm(formula = Churn ~ ., family = binomial(link = "logit"), data = training)
## 
## Deviance Residuals: 
##     Min       1Q   Median       3Q      Max  
## -2.0434  -0.6670  -0.2801   0.6633   3.1384  
## 
## Coefficients:
##                                      Estimate Std. Error z value Pr(>|z|)    
## (Intercept)                           0.27427    0.99878   0.275 0.783617    
## genderMale                           -0.01132    0.07803  -0.145 0.884628    
## SeniorCitizenYes                      0.23853    0.10109   2.360 0.018293 *  
## PartnerYes                           -0.09604    0.09300  -1.033 0.301758    
## DependentsYes                        -0.20339    0.10790  -1.885 0.059440 .  
## PhoneServiceYes                       1.19238    0.78903   1.511 0.130738    
## MultipleLinesYes                      0.62846    0.21490   2.925 0.003450 ** 
## InternetServiceFiber optic            2.97065    0.97062   3.061 0.002209 ** 
## InternetServiceNo                    -2.83480    0.98197  -2.887 0.003891 ** 
## OnlineSecurityYes                     0.01739    0.21717   0.080 0.936176    
## OnlineBackupYes                       0.19555    0.21237   0.921 0.357149    
## DeviceProtectionYes                   0.32051    0.21370   1.500 0.133664    
## TechSupportYes                        0.06997    0.21947   0.319 0.749869    
## StreamingTVYes                        1.09479    0.39709   2.757 0.005833 ** 
## StreamingMoviesYes                    1.15865    0.39599   2.926 0.003434 ** 
## ContractOne year                     -0.89811    0.13015  -6.900 5.19e-12 ***
## ContractTwo year                     -1.73998    0.21471  -8.104 5.32e-16 ***
## PaperlessBillingYes                   0.31589    0.08947   3.531 0.000415 ***
## PaymentMethodCredit card (automatic) -0.07260    0.13630  -0.533 0.594286    
## PaymentMethodElectronic check         0.32334    0.11302   2.861 0.004223 ** 
## PaymentMethodMailed check             0.05326    0.13669   0.390 0.696813    
## MonthlyCharges                       -0.08200    0.03859  -2.125 0.033579 *  
## tenure_group0-12 Month                1.54729    0.20287   7.627 2.40e-14 ***
## tenure_group12-24 Month               0.70617    0.19976   3.535 0.000408 ***
## tenure_group24-48 Month               0.40165    0.18209   2.206 0.027398 *  
## tenure_group48-60 Month               0.10277    0.19998   0.514 0.607306    
## ---
## Signif. codes:  0 '***' 0.001 '**' 0.01 '*' 0.05 '.' 0.1 ' ' 1
## 
## (Dispersion parameter for binomial family taken to be 1)
## 
##     Null deviance: 5702.8  on 4923  degrees of freedom
## Residual deviance: 4057.9  on 4898  degrees of freedom
## AIC: 4109.9
## 
## Number of Fisher Scoring iterations: 6
\end{verbatim}

Análise de recursos:

\begin{enumerate}
\def\labelenumi{\arabic{enumi}.}
\tightlist
\item
  Os três principais recursos mais relevantes incluem Contrato,
  Faturamento sem papel e grupo de posse, todos variáveis categóricas.
\end{enumerate}

\begin{verbatim}
## Analysis of Deviance Table
## 
## Model: binomial, link: logit
## 
## Response: Churn
## 
## Terms added sequentially (first to last)
## 
## 
##                  Df Deviance Resid. Df Resid. Dev  Pr(>Chi)    
## NULL                              4923     5702.8              
## gender            1     0.30      4922     5702.5 0.5862323    
## SeniorCitizen     1   117.94      4921     5584.5 < 2.2e-16 ***
## Partner           1   145.77      4920     5438.8 < 2.2e-16 ***
## Dependents        1    35.65      4919     5403.1 2.356e-09 ***
## PhoneService      1     2.07      4918     5401.0 0.1506076    
## MultipleLines     1     5.10      4917     5395.9 0.0238877 *  
## InternetService   2   476.55      4915     4919.4 < 2.2e-16 ***
## OnlineSecurity    1   168.49      4914     4750.9 < 2.2e-16 ***
## OnlineBackup      1    78.38      4913     4672.5 < 2.2e-16 ***
## DeviceProtection  1    50.77      4912     4621.7 1.039e-12 ***
## TechSupport       1    77.65      4911     4544.1 < 2.2e-16 ***
## StreamingTV       1     1.68      4910     4542.4 0.1951283    
## StreamingMovies   1     1.22      4909     4541.2 0.2698938    
## Contract          2   299.80      4907     4241.4 < 2.2e-16 ***
## PaperlessBilling  1    13.51      4906     4227.9 0.0002367 ***
## PaymentMethod     3    31.97      4903     4195.9 5.307e-07 ***
## MonthlyCharges    1     5.29      4902     4190.6 0.0214746 *  
## tenure_group      4   132.71      4898     4057.9 < 2.2e-16 ***
## ---
## Signif. codes:  0 '***' 0.001 '**' 0.01 '*' 0.05 '.' 0.1 ' ' 1
\end{verbatim}

Analisando a tabela de variância, podemos ver a queda no desvio ao
adicionar cada variável uma de cada vez. Adicionar InternetService,
Contract e tenure\_group reduz significativamente o desvio residual. As
outras variáveis, como PaymentMethod e Dependents, parecem melhorar
menos o modelo, embora todos tenham valores p baixos.

\hypertarget{avaliando-a-capacidade-preditiva-do-modelo}{%
\subsection{Avaliando a capacidade preditiva do
modelo}\label{avaliando-a-capacidade-preditiva-do-modelo}}

\begin{verbatim}
## [1] "Logistic Regression Accuracy 0.796489563567362"
\end{verbatim}

\hypertarget{confusion-matrix}{%
\subsection{Confusion Matrix}\label{confusion-matrix}}

\begin{verbatim}
## [1] "Confusion Matrix Para Logistic Regression"
\end{verbatim}

\begin{verbatim}
##    
##     FALSE TRUE
##   0  1393  155
##   1   274  286
\end{verbatim}

\hypertarget{odds-ratio}{%
\subsection{Odds Ratio}\label{odds-ratio}}

Uma das medidas de desempenho interessantes na regressão logística é
Odds Ratio. Basicamente, odds ratio é a chance de um evento acontecer.

\begin{verbatim}
##                                               OR       2.5 %      97.5 %
## (Intercept)                           1.31557456 0.185761078   9.3281343
## genderMale                            0.98874101 0.848498733   1.1521788
## SeniorCitizenYes                      1.26938732 1.041046799   1.5474295
## PartnerYes                            0.90842704 0.757031628   1.0901380
## DependentsYes                         0.81595914 0.659860605   1.0074369
## PhoneServiceYes                       3.29491284 0.702762810  15.5040351
## MultipleLinesYes                      1.87472756 1.231037896   2.8590313
## InternetServiceFiber optic           19.50461870 2.920473588 131.3167605
## InternetServiceNo                     0.05873012 0.008542775   0.4016022
## OnlineSecurityYes                     1.01754257 0.664632442   1.5574239
## OnlineBackupYes                       1.21598004 0.802101967   1.8444538
## DeviceProtectionYes                   1.37782546 0.906637252   2.0957710
## TechSupportYes                        1.07247562 0.697371292   1.6489467
## StreamingTVYes                        2.98854754 1.374086825   6.5196678
## StreamingMoviesYes                    3.18563460 1.467946970   6.9350008
## ContractOne year                      0.40733914 0.314582671   0.5241617
## ContractTwo year                      0.17552435 0.113643993   0.2641466
## PaperlessBillingYes                   1.37148583 1.151205659   1.6349813
## PaymentMethodCredit card (automatic)  0.92997326 0.711664577   1.2146054
## PaymentMethodElectronic check         1.38173569 1.108034870   1.7259887
## PaymentMethodMailed check             1.05470075 0.807190118   1.3796318
## MonthlyCharges                        0.92127300 0.854042899   0.9935463
## tenure_group0-12 Month                4.69874062 3.167955260   7.0206728
## tenure_group12-24 Month               2.02622597 1.373371827   3.0069142
## tenure_group24-48 Month               1.49428414 1.048772145   2.1428426
## tenure_group48-60 Month               1.10824093 0.748845393   1.6414995
\end{verbatim}

Para cada aumento de unidade no encargo mensal (Monthly Charge), há uma
redução de 2,5\% na probabilidade do cliente cancelar a assinatura.

\hypertarget{decision-tree}{%
\subsection{Decision Tree}\label{decision-tree}}

Para fins de ilustração, vamos usar apenas três variáveis para plotar
árvores de decisão, elas são ``Contrato'', ``tenure\_group'' e
``PaperlessBilling''.

\includegraphics{07-Big-Data-na-Pratica4_files/figure-latex/unnamed-chunk-34-1.pdf}

\begin{enumerate}
\def\labelenumi{\arabic{enumi}.}
\tightlist
\item
  Das três variáveis que usamos, o Contrato é a variável mais importante
  para prever a rotatividade de clientes ou não.
\item
  Se um cliente em um contrato de um ano ou de dois anos, não importa se
  ele (ela) tem ou não a PapelessBilling, ele (ela) é menos propenso a
  se cancelar a assinatura.
\item
  Por outro lado, se um cliente estiver em um contrato mensal, e no
  grupo de posse de 0 a 12 meses, e usando o PaperlessBilling, esse
  cliente terá mais chances de cancelar a assinatura.
\end{enumerate}

\begin{verbatim}
## [1] "Confusion Matrix for Decision Tree"
\end{verbatim}

\begin{verbatim}
##          Actual
## Predicted   No  Yes
##       No  1398  342
##       Yes  150  218
\end{verbatim}

\begin{verbatim}
## [1] "Decision Tree Accuracy 0.766603415559772"
\end{verbatim}

\hypertarget{random-forest}{%
\subsection{Random Forest}\label{random-forest}}

\begin{verbatim}
## 
## Call:
##  randomForest(formula = Churn ~ ., data = training) 
##                Type of random forest: classification
##                      Number of trees: 500
## No. of variables tried at each split: 4
## 
##         OOB estimate of  error rate: 21.24%
## Confusion matrix:
##       No Yes class.error
## No  3255 360  0.09958506
## Yes  686 623  0.52406417
\end{verbatim}

A previsão é muito boa ao prever ``Não''. A taxa de erros é muito maior
quando se prevê ``sim''.

\hypertarget{prediction-e-confusion-matrix}{%
\subsection{Prediction e confusion
matrix}\label{prediction-e-confusion-matrix}}

\begin{verbatim}
## Confusion Matrix and Statistics
## 
##           Reference
## Prediction   No  Yes
##        No  1398  294
##        Yes  150  266
##                                           
##                Accuracy : 0.7894          
##                  95% CI : (0.7713, 0.8066)
##     No Information Rate : 0.7343          
##     P-Value [Acc > NIR] : 2.734e-09       
##                                           
##                   Kappa : 0.4119          
##                                           
##  Mcnemar's Test P-Value : 1.149e-11       
##                                           
##             Sensitivity : 0.9031          
##             Specificity : 0.4750          
##          Pos Pred Value : 0.8262          
##          Neg Pred Value : 0.6394          
##              Prevalence : 0.7343          
##          Detection Rate : 0.6632          
##    Detection Prevalence : 0.8027          
##       Balanced Accuracy : 0.6891          
##                                           
##        'Positive' Class : No              
## 
\end{verbatim}

\hypertarget{taxa-de-erro-para-o-modelo-de-floresta-aleatuxf3rio}{%
\subsection{Taxa de erro para o modelo de floresta
aleatório}\label{taxa-de-erro-para-o-modelo-de-floresta-aleatuxf3rio}}

\includegraphics{07-Big-Data-na-Pratica4_files/figure-latex/unnamed-chunk-40-1.pdf}

\begin{verbatim}
## mtry = 4  OOB error = 21.79% 
## Searching left ...
## mtry = 8     OOB error = 22.73% 
## -0.04287046 0.05 
## Searching right ...
## mtry = 2     OOB error = 21% 
## 0.03634669 0.05
\end{verbatim}

\includegraphics{07-Big-Data-na-Pratica4_files/figure-latex/unnamed-chunk-41-1.pdf}

\hypertarget{ajustar-o-modelo-de-floresta-aleatuxf3rio-novamente}{%
\subsection{Ajustar o modelo de floresta aleatório
novamente}\label{ajustar-o-modelo-de-floresta-aleatuxf3rio-novamente}}

\begin{verbatim}
## 
## Call:
##  randomForest(formula = Churn ~ ., data = training, ntree = 200,      mtry = 2, importance = TRUE, proximity = TRUE) 
##                Type of random forest: classification
##                      Number of trees: 200
## No. of variables tried at each split: 2
## 
##         OOB estimate of  error rate: 20.71%
## Confusion matrix:
##       No Yes class.error
## No  3311 304  0.08409405
## Yes  716 593  0.54698243
\end{verbatim}

\hypertarget{torne-as-previsuxf5es-e-a-matriz-de-confusuxe3o-novamente}{%
\subsection{Torne as previsões e a matriz de confusão
novamente}\label{torne-as-previsuxf5es-e-a-matriz-de-confusuxe3o-novamente}}

\begin{verbatim}
## Confusion Matrix and Statistics
## 
##           Reference
## Prediction   No  Yes
##        No  1426  302
##        Yes  122  258
##                                           
##                Accuracy : 0.7989          
##                  95% CI : (0.7811, 0.8158)
##     No Information Rate : 0.7343          
##     P-Value [Acc > NIR] : 3.082e-12       
##                                           
##                   Kappa : 0.4256          
##                                           
##  Mcnemar's Test P-Value : < 2.2e-16       
##                                           
##             Sensitivity : 0.9212          
##             Specificity : 0.4607          
##          Pos Pred Value : 0.8252          
##          Neg Pred Value : 0.6789          
##              Prevalence : 0.7343          
##          Detection Rate : 0.6765          
##    Detection Prevalence : 0.8197          
##       Balanced Accuracy : 0.6910          
##                                           
##        'Positive' Class : No              
## 
\end{verbatim}

\hypertarget{random-forest-feature-importance}{%
\subsection{Random Forest Feature
Importance}\label{random-forest-feature-importance}}

\includegraphics{07-Big-Data-na-Pratica4_files/figure-latex/unnamed-chunk-44-1.pdf}

\end{document}
